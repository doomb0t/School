
\documentclass[10pt,letterpaper, cm]{hmcpset}
\usepackage[top=0.5in,bottom=1in,right=1in,left=0.75in]{geometry}
\usepackage{multicol,graphicx, enumerate, cancel, amsmath, amsthm}
\usepackage{tikz}
\usepackage{tikz,fullpage}

\usetikzlibrary{arrows,petri,topaths,shapes, backgrounds}
				
\usepackage{tkz-berge}
\usepackage{tkz-graph}

\name{Jonathon Sonesen}
\class{CS 251}
\duedate{\today}
\assignment{Final Exam (Of Doom)}

\setlength{\columnsep}{0.5cm}
\setlength{\columnseprule}{00pt}
\begin{document}
\begin{problem}[1]
State presisiley (but concisely) in your own words what it mean for an argument to be valid.
\end{problem}\\
\\An argument is considered valid if it's premise and conclusion are logically sound. \\

\begin{problem}[2]
  Prove (or disprove).\\
  \begin{center}
    \begin{equation*}
      \forall n \in \mathbb{Z}, n \geq 3 \rightarrow \sum\limits_{i=3}^{n}i(i-1) 
      =  \frac{(n-2)(n^2+2n+3)}{2}
    \end{equation*}
  \end{center}
\end{problem}\\

\\
Suppose:
\begin{center}
  \begin{equation*}
    \exists n \in \mathbb{Z}, n \geq 3 \land  \sum\limits_{i=3}^{n}i(i-1) 
      \neq \frac{(n-2)(n^2+2n+3)}{2}
  \end{equation*}
\end{center}
\begin{center}
\begin{array}{lll}
  Row & Statement & Comment\\
  1.& Let~n:= 3&\text{counter~example}\\
  2.& \sum\limits_{i=3}^{3}i(i-1) = 6 &\\
  3.&\frac{(3-2)(3^2+2 \cdot 3+3)}{2} = 9 & \\
  4.&6 \neq 9 &\\
  5.& \therefore \exists n \in \mathbb{Z}, n \geq 3 \land  \sum\limits_{i=3}^{n}i(i-1) 
  \neq \frac{(n-2)(n^2+2n+3)}{2} & \text{QED}\\
\end{array}
\end{center}

\begin{problem}[3]
  Prove (or disprove) that for all sets A, B, and C\\
  \begin{equation*}
    A \cup (B \cap C) \subset (A \cup B) \cap C
  \end{equation*}
\end{problem}\\
\\
\begin{problem}[4]
  Let $F:\mathbb{R}~x~\mathbb{R} \rightarrow \mathbb{R}~x~\mathbb{R}$ 
  be a relation defined as\\
  \begin{equation*}
    F(x,y) = (3y-1,1-x)
  \end{equation*}
  \begin{center}
    \begin{enumerate}[(a)]
      \item Prove or disprove that F is a function.\\
      \item Prove (or disprove) $ F^{-1}$ is a function\\
    \end{enumerate}
  \end{center}
\end{problem}

\begin{problem}[5]
Let: \\
\begin{center}
  \begin{equation*}
    A= \left\{ x \in \mathbb{Z} \mid \exists~k \in \mathbb{Z} \land x = k^2 \right\}
  \end{equation*}
\end{center}
What is $\mid A \mid $? Explain your reasoning and justify your answer.
\end{problem}

\begin{problem}[6]
Define A to be the set of unique digits on your PCC G-Number, and let R :$A\rightarrow A$ be\\
deifined as\\
\begin{center}
  \begin{equation*}
    xRy \leftrightarrow 2\mid(x-y)
  \end{equation*}
\end{center}\\
List the equivalence classes of R or prove no such classes exist.
\end{problem}

\begin{problem}[7]
  Let S be the set of all strings of 0's and 1's of length 3. Define $ R:S \rightarrow S$ as\\ 
  \begin{center}
    \begin{align*}
      ~the~two~left~most~characters& \\
      sRt \leftrightarrow~
      of~s~are~the~same~as~the~two\\
      left~most~characters~of~t 
    \end{align*}
  \end{center}
  List the equivalence classes of R or prove np such classes exist.
\end{problem}\\

\begin{problem}[8]
  Let $x$ be you PCC G number without the leading G. What is the inverse modulo of
  x modulo 8831?
\end{problem}\\

\begin{problem}[9]
  An RSE Cipher has the public key pq=65 and e=7. What is the encrypted value of the
  last 3 digits of your PCC G number?
\end{problem}\\

\begin{problem}[10]
  Three quizzes are given to a class of 30 students, and all student submitted all quizzes.
  Given:\\
  \begin{enumerate}[-]
    \item 15 students scored 12 or more on quiz 1
    \item 12 students scored 12 or more on quiz 2
    \item 18 students scored 12 or more on quiz 3
    \item 7 students scored 12 or more on quizzes 1 and 2
    \item 11 students scored 12 or more on quizzes 1 and 3
    \item 8 students scored 12 or more on quizzes 2 and 3
    \item 4 students scored 12 or more on quizzes 1, 2, and ,3
  \end{enumerate}
  How many students scored 12 or more on quizzes 1 and 2 but not 3?
\end{problem}\\

\begin{problem}[11]
  An urn contains four balls, each numbered with one of the last 4 digits
  of your PCC G-Number. If a person selects two balls at random (equal
  probability), what is the expected value of the product of the numbers on
  the balls?
\end{problem}\\

\begin{problem}[12]
   If a graph has nodes of degrees 1, 1, 2, 3, and 3, how many edges does it
   have? Explain your reasoning and justify your answer, although a formal
   proof is not needed.
\end{problem}\\

\begin{problem}[13]
   Specify (drawing or adjacency matrix) a full binary tree with 16 nodes, of
   which 6 are internal nodes, or prove no such graph exists.   
\end{problem}\\

\begin{problem}[14]
  Must a graph with 68 nodes and 72 edges have a circuit? Explain your
  reasoning and justify your answer, although a formal proof is not needed.  
\end{problem}\\

\begin{problem}[15]
  
\end{problem}\\

\end{document}
