\documentclass{article}
\usepackage[utf8]{inputenc}

\title{CS250 Methods Of Proof Knowledge Test}
\author{jonathon.sonesen }
\date{February 2015}

\begin{document}

\maketitle
$
\section{\forall x \in \mathbb{R}, ~x > 1 ~\rightarrow~ x^2 > x}
\begin{array}{l l l}
Row & Statement & Comment \\
1 & x \in \mathbb{Z} ~\mid~ ~x > 1   & \text{generic~particular} \\
2 & x^2 ~=~ x \cdot x & \text{multiplication} \\
3 & x ~=~ x \cdot 1& \text{multiplication} \mathbb{R}\\
4 & x \cdot x > ~x \cdot 1& \text{by~multiplication} \\
5 & \therefore \forall x \in \mathbb{R}, ~x > 1 ~\rightarrow~ x^2 > x & \text{QED} \\
\end{array}

$
\section{\exists x \in \mathbb{R} ~\mid~ 0 < x < 1 ~\wedge~ x^2 \geq x}
\begin{array}{l l l}
Row & Statement & Comment\\
1 & Suppose:  \forall x \in \mathbb{R},  0 < x < 1 ~\rightarrow~ x^2 ~<~ x  & \text{negation} \\
2 &  \forall x \in \mathbb{R} ~\mid~ (0 < x < 1) & \text{generic~particular} \\
3 & \forall x \in \mathbb{R}, \exists y \in \mathbb{R} ~|~ y > 1 \wedge x~=~ \frac{1}{y} & \text{by~ definition} \\
4 & ~(\frac{1}{y})^2 < \frac{1}{y} & \text{by~substitution}\\
5 &  \frac{1}{y^2} < \frac{1}{y} & \text{multiplication~of~fractions~T15} \\
6 &\therefore  \forall x \in \mathbb{R},  0 < x < 1 ~\rightarrow~ x^2 ~<~ x & \text{QED} \\
\end{array}

$
\section{\forall n \in \mathbb{Z}, ~n>0 ~\rightarrow~ 2^{2^n} ~+~ 1 ~\in \mathbb{Z}^{P}}
\begin{array}{l l l}
Row & Statement & Comment\\
1 & Suppose: \exists n \in \mathbb{Z}, ~n>0 \wedge 2^{2^n} ~+~ 1 ~ \in \mathbb{{Z}^{C}} & \text{negation} \\
2 & Let: n ~:=~ 5 & \text{counter example} \\
3 & 2^{2^5} ~+~ 1 ~=~ 4,294,967,297 = (641) \cdot (6700417) & \text{by~ definition} \\
4 & \therefore \exists n \in \mathbb{Z}, ~n>0 \wedge 2^{2^n} ~+~ 1 ~ \in \mathbb{{Z}^{C}}  & \text{QED}\\
\end{array}

$https://www.sharelatex.com/project
\section{\forall a,b,c,d \in \mathbb{Z},~x \in \mathbb{R},\left ( ~a \neq c ~\wedge~ \left ( \frac{ax~+~b}{cx~+~d} ~=~ 1 \right ) \right ) ~\rightarrow~ x \in \mathbb{Q}}
\begin{array}{l l l}
Row & Statement & Comment\\
1 &  \forall a,b,c,d \in \mathbb{Z},~x \in \mathbb{R},\left ( ~a \neq c ~\wedge~ \left ( \frac{ax~+~b}{cx~+~d} ~=~ 1 \right )& \text{generic~particular} \\
2 &  \forall a,b,c,d \in \mathbb{Z},~x \in \mathbb{R},\left ax + b = cx + d & \text{cross~multiply} \\
3 &ax - cx = d - b  & \text{algebra} \\
4 &x(a-c) = d-b  & \text{by factorization}\\
5 &x = \frac{d-b}{a-c}, a \neq c & \text{algebra}\\
6 & x \in \mathbb{Q} &\text{definition}\\
7 & \exists y \int \mathbb{Z}| y = d - b & \text{by closure}\\
8 & \exists w \int \mathbb{Z}| w = a - c \neq 0 & \text{by closure`doesnt~equal~zero~a \neq c} \\
9 & x = \frac{y}{w}     &\text{ratio~of~two~ints} \\
10 & \therefore \forall a,b,c,d \in \mathbb{Z},~x \in \mathbb{R},\left ( ~a \neq c ~\wedge~ \left ( \frac{ax~+~b}{cx~+~d} ~=~ 1 \right ) \right ) ~\rightarrow~ x \in \mathbb{Q}& \text{QED}
\end{array}

$
\section{\forall x \in \mathbb{R}, ~x \in \mathbb{Q} ~\rightarrow~ 5x^3 ~+~ 8x^2 ~+~ 7 ~\in \mathbb{Q}}
\begin{array}{l l l}
Row & Statement & Comment\\
1 & Suppose: \exists ~c \in \mathbb{R} \mid c = 5x^3 + 8x^2 + 7 & \text{generic~particular} \\
2 & \forall x \in \mathbb{Q} , \exists a, b \in \mathbb{Z} ~\mid~ x = \frac{a}{b}& \text{by~definition~of~rational~num} \\
3 &5x^3 + 8x^2 + 7 = c = \frac{5a^3}{b^3} + \frac{8a^2}{b^2} + \frac{7}{1}  & \text{by~substitution} \\
4 & c = \frac{5a^3b^2 + 8a^2b^3+7b^3}{b^3 \cdot b^2} & \text{by~cross~multiply}\\
5 & c = \frac{5a^3 + 8a^2b^3+7b^3}{b^3}& \text{by~division~of~b^2}\\
6 & Let: y = 5a^3 + 8a^2b^3+7b^3& \text{by~substitution}\\
7 & y \in \mathbb{Z} & \text{by~closure~on~sum~and~products~on~int}\\
8 &  Let: z = b^3 &\text{by~substitution} \\
9 & y \in \mathbb{Z} &\text{by~closure~of~products} \\
10&c = \frac{y}{z} & \text{by~substitution}\\
11& c \in \mathbb{Q} & \text{by~closure }\\
12 &\forall x \in \mathbb{R}, ~x \in \mathbb{Q} ~\rightarrow~ 5x^3 ~+~ 8x^2 ~+~ 7 ~\in \mathbb{Q}& \text{QED}
\end{array}

$
\section{\exists a,b,c \in\mathbb{Z} ~\mid~ a \mid bc \wedge a \nmid b ~\wedge~ a \nmid c}
\begin{array}{l l l}
Row & Statement & Comment\\
1 & Let: a := 4 & \text{example} \\
2 & Let: b := 2 & \text{example} \\
3 & Let: c := 6 & \text{example} \\
4 & 4 | 12 & \text{substitution}\\
5 & 4 \nmid 2  & \text{substitution}\\
6 &  4 \nmid 6 & \text{by~substitution}\\
7 &\therefore \exists a,b,c \in\mathbb{Z} ~\mid~ a \mid bc \wedge a \nmid b ~\wedge~ a \nmid c & \text{QED}
\end{array}

$
\section{\forall a,n \in \mathbb{Z}, ~a \mid n^2 ~\wedge~ a \leq n ~\rightarrow~ a \mid n}
\begin{array}{l l l}
Row & Statement & Comment\\
1 & \exists a,n \in \mathbb{Z}, a \mid n^2 \wedge a \leq n \wedge a \nmid n & \text{negation for counter example} \\
2 & Let: a := 4 & \text{example} \\
3 & Let: n := 6 & \text{example} \\
4 & 4 | 6 ^2 \wedge 4 \leq 6 \wedge 4 \nmid 6 & \text{substitution}\\
5 & 6 ^2 = 36 & \text{by multiplication}\\
6 & 9 \cdot 4 = 36 & \text{by multiplication}\\
7 &\exists a,n \in \mathbb{Z}, a \mid n^2 \wedge a \leq n \wedge a \nmid n  & \text{QED}
\end{array}


$
\section{\forall m,n \in \mathbb{Z}, ~ m \mod 5 ~=~ 2 ~\wedge~ n \mod 5 ~=~ 1 ~\rightarrow~ nm \mod 5 ~=~ 1}
\begin{array}{l l l}
Row & Statement & Comment\\
1 & \exists m,n \in \mathbb{Z}, ~ m \mod 5 ~=~ 2 ~\wedge~ n \mod 5 ~=1 ~\wedge~ nm \mod 5 \neq 1& \text{counter~example}\\
2 & Let: m := 7& \text{set m to 7}\\
3 & Let: n:= 6 & \text{set n to 6}\\
4 & 7 \mod 5 = 2& \text{by math}\\
5 & 6 \mod 5 = 1& \text{by math}\\
6 & 7 \cdot 6 = 42&\\
7 & 42 \mod 5 = 2&\\
8 & \therefore \exists m,n \in \mathbb{Z}, ~ m \mod 5 ~=~ 2 ~\wedge~ n \mod 5 ~=1 ~\wedge~ nm \mod 5 \neq 1& \text{QED}\\
\end{array}

$
\section{\forall m,d,k \in \mathbb{Z}, ~d > 0 ~\rightarrow~ (m ~+~ dk) \mod d ~=~ m \mod d}
\begin{array}{l l l}
Row & Statement & Comment\\
1 & \forall m,d,k \in \mathbb{Z}, ~d > 0 ~\rightarrow~ (m ~+~ dk) \mod d ~=~ m \mod d& \text{theorem}\\
2 &\forall m,d,k \in \mathbb{Z}, ~d > 0& \text{generic}\\
3 & Let:\exists x \in \mathbb{Z}, x =(m+dk) \mod d& \text{by substitution}\\
4 & Let: \exists y \in \mathbb{Z} ~\mid~ m + dk = yd + x & \text{by~quotient~remainder~theorem}\\
5 & m = (y-k)d + x & \text{by~quotient remainder~r~is~remainder}\\
6 & m \mod d = r &\\
7 &\therefore   \forall m,d,k \in \mathbb{Z}, ~d > 0 ~\rightarrow~ (m ~+~ dk) \mod d ~=~ m \mod d& \text{QED}\\
\end{array}

$
\section{\forall n \in \mathbb{Z}, ~n \equiv 1 ( \mod 2) ~\rightarrow~ \lceil \frac{n^2}{4} \rceil ~=~ \frac{n^2 ~+~ 3}{4}}
\begin{array}{l l l}
Row & Statement & Comment\\
1 &Let: \forall n \in \mathbb{Z} \exists r \in \mathbb{Z}, n = 2k+1 & \text{definition~of~odd~int,generic~particular}\\
2 &  \lceil \frac{n^2}{4} \rceil = \lceil \frac{(2k+1)^2}{4} \rceil& \text{substitution}\\
3 &\lceil \frac{(2k+1)^2}{4} \rceil = \lceil \frac{(4k^2 + 1 + 4k}{4} \rceil & \text{algebra}\\
4 & \lceil \frac{(2k+1)^2}{4} \rceil = \lceil \frac{(4k^2 + 4k}{4} + \frac{1}{4}\rceil& \text{algebra}\\
5 &  \lceil \frac{(4k^2 + 4k}{4} + \frac{1}{4}\rceil = \lceil \frac{(4(k^2 + k}{4} + \frac{1}{4}\rceil& \text{by~math}\\
6 & \lceil \frac{(4(k^2 + k}{4} + \frac{1}{4}\rceil =k^2 + k + 1&\\
7 & \frac{n^2 + 3}{4} = \frac{(2k+1)^2 + 3}{4}& \text{substitution~for~right~side}\\
8 & \frac{(2k+1)^2 + 3}{4} = \frac{4k^2 +4k + 4}{4} & \text{expand}\\
9 & \frac{4k^2 +4k + 4}{4} = \frac{4(k^2 + k + 1}{4} & \text{factor}\\
10 & \frac{4(k^2 + k + 1}{4}  = k^2 +k + 1& \text{division}\\
11 & \therefore \forall n \in \mathbb{Z}, ~n \equiv 1 ( \mod 2) ~\rightarrow~ \lceil \frac{n^2}{4} \rceil ~=~ \frac{n^2 ~+~ 3}{4} & \text{QED}
\end{array}

$
\section{\forall x \in \mathbb{R}, ~\lfloor x^2 \rfloor ~=~ \lfloor x \rfloor^2}
\begin{array}{l l l}
Row & Statement & Comment\\
1 &\exists x \in \mathbb{R}, \lfloor x^2 \rfloor \neq \lfloor x \rfloor ^2 & \text{counter example}\\
2 &Suppose: \exists x \in \mathbb{R}, x := 1.5 & \text{counter example}\\
3 &\lfloor 1.5^2 \rfloor = 2  & \text{}\\
4 & \lfloor 1.5 \rfloor = 1& \text{}\\
5 & 1 ^2 = 1& \text{}\\
6 &\therefore \exists x \in \mathbb{R}, \lfloor x^2 \rfloor \neq \lfloor x \rfloor ^2&\\
\end{array}

$
\section{\forall n \in \mathbb{Z}, ~n \mod 6 ~=~ 3 ~\rightarrow~ n \mod 3 ~\neq~ 2}
\begin{array}{l l l}
Row & Statement & Comment\\
1 &Assume: \exists a \in \mathbb{Z}, a \mod 6 = 3 \wedge a \mod 3 = 2 & \text{proof~by~contradiction}\\
2 & \exists p, k ~\mid~ a = 3p +2 = 6k + 3 & \text{quotient remainder}\\
3 & 3(p + 2k) = 1   & \text{algebra}\\
4 &  p + 2k = \frac{1}{3}& \text{always divides by three, and is a contradiction}\\
5 & \therefore \forall n \in \mathbb{Z}, ~n \mod 6 ~=~ 3 ~\rightarrow~ n \mod 3 ~\neq~ 2 & \text{}\\
\end{array}


\section{\forall a,b \in \mathbb{Z}, ~gcd(a,b) \mid lcm(a,b)}
\begin{array}{l l l}
Row & Statement & Comment\\
1 & Let: \forall a, b \in \matbb{Z}, \exists x, y \in \mathbb{Z} ~\mid~ x = ~gcd(a,b) \wedge y = ~lcm(a,b)  & \text{}\\
2 & x|a, x|b ~\wedge~ a|y, b|y & \text{quotient remainder}\\
3 & \forall x \in \mathbb{Z}, \exists p, q \in \mathbb{Z} ~\mid~ a = x\cdot p \wedge b  = x\cdot q & \text{algebra}\\
4 &\forall y \in \mathbbPZ}, \exists r, s \in \mathbb{Z} ~\mid~ y = a \cdot r \wedge y = b \cdot s & \text{}\\
5 & y = a \cdot r & \text{}\\
6 & y = r \cdot x \cdot p &  \text{by~substitution}\\
7 & y = x(rp) & \text{commutative}\\
8 & lcm(ab) = gcm(ab) \cdot rp & \text{}\\
9 & lcm(ab) | gcm(ab) = rp & 
\end{array}

A=\{ x \in \mathbb{Z}| \frac{3x^3+x^2-2x+4}{3x+4}| \leq (2^{50} -1) \}
\end{document}